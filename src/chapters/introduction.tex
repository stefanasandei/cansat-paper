\subsection{Purpose of the mission}

The main goal of this mission will be the creation of a rocket that will resist launch and landing, that will have a usable and safe recovery system, and that will be able to gather information all through its travel time and send it back so that it can be analysed. The information such as temperature and altitude measured, after being analysed, will be able to be used in order to understand how these markers change as we get higher up, especially in a populated city, where pollution would inevitably make the air heavier and warmer closer to the surface. 

\subsection{Team organisation and roles}

We are team Lunateeks, a name that was chosen by probably the most creative person out of all of us, Stefan, with the aim of describing the competition we are participating in, but also us, as we are a bit overly passionate about physics and science in general. We have created a mixed team, half of us being from the National College of Iasi, and the other half from "Emil Racovita" National College, and we have met each other through our aforementioned love for basically anything regarding science.

All of us are part of The Center of Excellence for Physics, which is one of the places where we usually get together and discuss what we should do and how, and we are all attending classes in the mathematical programming profile. As for prior experience, we do not have much, aside from participating in the ROSPIN School, which prepared us for the competition, but we have participated in different activities that involved robotics and the launch of water rockets. All of us also want to continue with science further on in our lives, Stefan wanting to keep on with his passion for programming by going into deep learning research, Bianca choosing to probably follow the path of microengineering, and Ilinca the one of physics or forensic science. While Anemona and Tudor are not 100\% sure what they want to be doing, they know it will be something in the field of science. 

Bianca is our team leader because she was the one to bring us all together, Anemona and Ilinca are the team's propulsion engineers because they always think outside the box and come up with unique ideas, Stefan is our Software Engineer, as he can understand and work with basically everything, especially lines of code, and lastly, Tudor is our Electrician because he simply passionate about this field.  Anemona, alongside Stefan and Bianca, were the ones who worked the most on the technical aspect of this rocket, developing the ideas discussed by all of us and finding the best solutions that could actually be applied. Tudor worked on the little things, specifically focusing on the parts of how the rocket would work in practice and not just in theory. Ilinca was the one that worked on "the charm", or "the bluffing" as she likes to call it, bringing everything together and focusing on the introductions and conclusions that marked the ending and the beginning of this report.

\subsection{Mission objectives}

Our mission objectives are to design a rocket that would meet the following conditions:

\begin{itemize}
  \item The rocket must survive the launch and the landing successfully, without damaging the electrical parts
  \item The rocket should ascend to a height of approximately 256m 
  \item The payload will be detach from the rocket’s body successfully, at the maximum height, both parts remaining intact and functional
  \item Both parachutes, for the payload and for rocket, must deploy and slow their descent, therefore reaching a maximum speed of 64.3 $ \frac{m}{s} $ and hitting the ground with a speed of 7.07 $ \frac{m}{s} $
  \item The sensors will record information about temperature, altitude, pressure, and acceleration so that the information can be further centralised and conclusions can be drawn
  \item We will use the information to measure the rarefaction of the air (number of molecules per volume unit) in accordance to the altitude, and also use said graph to predict the level of rarefaction for higher altitudes
  \item With the information from the GPS, a graph of the rocket’s trajectory will be created using Google Earth
\end{itemize}
