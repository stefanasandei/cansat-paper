\subsection{Data Analysis Plan}

The Data Analysis phase will happen concurrently with the CanSat’s flight. The ground station laptop, which acts as a receiver, will listen to incoming packets from the CanSat and forward them to the data analysis software. The program consists of a Python3 web server which will render live mission data to a webpage on localhost. We will be using an HTTP Tunnel (using ngrok) to forward that webpage to multiple devices in different networks. The graphs will be rendered using the Matplotlib library and the ML-related tasks will be done using the scikit-learn library. This software also has the job of serialising the data to local files so they can be later viewed. It will also support writing out the 3D coordinates in a specific XML format such as it can be viewed in Google Earth. The specific data analysis will be done using machine learning algorithms, provided by the scikit-learn library.

\subsection{Outreach Program}

The Outreach Program that will educate the population about our mission is made to target pupils and students, as well as other participants to upper forms of education. 

\begin{itemize}

\item \textbf{Presentations}: Our team will hold presentations in the members’ schools (National College Iasi \& National College “Emil Racovita” Iasi), as well as to the astronomical communities in Iasi: The Astroclub and Iasi Space Network. These will contain information about the materials and the technologies used, as the public will be highly educated and interested in these topics.
\item \textbf{Workshops}: With the help of the previously mentioned organizations and communities, as well as the robotics clubs near us (Cyliis and Peppers), we can organize various meetings and hold workshops. Many students in our city are interested in learning how to build the Rocket, as well as the electronic parts of the payload.
\item \textbf{Demonstrations}: The environmental tests can be repeated with an audience to create interest in our team and the competition. The documentation from the 3 tests will be uploaded to various platforms, such as YouTube, Facebook, etc. (more on the social media section below)
\item \textbf{Social media}: Our team will share the experiences we created during this competition on various forms of social media, such as Instagram, TikTok, Facebook, Youtube, and X(formerly Twitter). Those platforms cater to different age groups. Because any advertising is good advertising, we choose to create content for everyone, from children to elders.
\item \textbf{Websites}: Media that record our progress, as well as information on presentations, workshops, demonstrations, social media, community engagement, and science fairs and events, will be uploaded on the team’s website.
\item \textbf{Community Engagement}: Our team will visit local schools, such as the ones already mentioned, as well as the “Grigore C. Moisil” Computer Science High School of Iasi, National College Costache Negruzzi Iasi, etc. We also target any school and community center that presents interest in these domains. 
\item \textbf{Science Fairs or events}: Our team will participate in the science fairs that are held in our city to spread the knowledge of the existence of our team, what we do, and what we hope to achieve. 

\end{itemize}