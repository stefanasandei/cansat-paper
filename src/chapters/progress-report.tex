\subsection{New progress statement for the team profile}

After splitting all the tasks between us, according to what our strong suits are, all of us worked on the PDR and managed to send it, before the deadline, with almost nothing left unfinished, only some unclarities present. 

That being said, working on the CDR and finishing it was pretty much straightforward, as we all took the task we already had, improved the content and the ideas, and answered the feedback we were given. For the new tasks that appeared in the CDR, we simply split them up just like the ones from the PDR.

\subsection{Tasks list}

We are very pleased to say that, just like our PDR, all of our tasks have been finished before the deadline, therefore being able to send a complete CDR.

\newcommand{\done}{\textbf{\textcolor{green!50!black}{Done}}}

\begin{enumerate}[label=\arabic*.,leftmargin=*]
    \item Progress Report - \done
    \begin{enumerate}[label=\arabic*.]
        \item New progress statement for the team profile
        \item Tasks list
        \item Detailed project status
    \end{enumerate}
    \item Introduction - \done
    \begin{enumerate}[label=\arabic*.]
        \item Purpose of the mission
        \item Team organisation and roles
        \item Mission objectives
    \end{enumerate}
    \item Payload description - \done
    \begin{enumerate}[label=\arabic*.]
        \item Mission Overview
        \item Mechanical/structural design
        \begin{enumerate}[label=\arabic*.]
            \item Mechanical Design
            \item Components
            \item Placement
            \item Explanation
        \end{enumerate}
        \item Electrical design
        \begin{enumerate}[label=\arabic*.]
            \item Electrical Interface
            \item Radio Communication
            \item Power Consumption
        \end{enumerate}
        \item Software design
        \begin{enumerate}[label=\arabic*.]
            \item Software Program Flow
            \item Data Gathering and Storage
            \item Development Environment
        \end{enumerate}
        \item Recovery system
        \item Ground support equipment
    \end{enumerate}
    \item Rocket description - \done
    \item Project planning - \done
    \begin{enumerate}[label=\arabic*.]
        \item Time schedule of the Payload preparation
        \item Resource estimation
        \begin{enumerate}[label=\arabic*.]
            \item Budget
            \item External support
        \end{enumerate}
        \item Test plan
        \item Time management
    \end{enumerate}
    \item Data analysis and outreach - \done
    \begin{enumerate}[label=\arabic*.]
        \item Data Analysis Plan
        \item Outreach Program
    \end{enumerate}
    \item Conclusion - \done
    \begin{enumerate}[label=\arabic*.]
        \item Summary of the CDR
        \item Recommendations for next steps
    \end{enumerate}
\end{enumerate}

\subsection{Detailed project status}

While our CDR is complete and well-worked, we have encountered some setbacks during this period. 

One thing that we are still not sure about is whether there will be interference in the radio communications between us and the rocket, as it could be caused by the speed the rocket is going or the antenna that is inside the payload. We still intend to continue our research on this and figure out if the interference could actually happen, and, if yes, how we could resolve this issue.

Another issue we faced was trying to figure out how to detach the payload from the rocket, as it was one of the things we overlooked during the work on the PDR. In the end, after a generous amount of research done by all of us, we found a way how this can be done. Seeing that the engine had a delay, we thought to use the last remaining power to create high pressure in the body tube so that the payload would detach because of the force created inside.

We also managed to solve another problem we had in the PDR, that being the maximum height the rocket would ascend to. While we tried to use lighter materials, that did not do much, so our solution was to add not one, but two Aerotech D9-5 engines, therefore making it possible for the rocket to reach approximately 256m.
